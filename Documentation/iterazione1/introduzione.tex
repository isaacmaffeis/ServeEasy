\section{Introduzione}
Nella Iterazione 1 si sono presi i casi d'uso a più alta priorità e ci si è focalizzati allo sviluppo della architettura software, del database e dell'algoritmo. 
Si è adottato un approccio di good design, puntando a un sistema software di alta qualità, mantenibile e scalabile, con componenti modulari e codice chiaro. Parallelamente, si è perseguito il principio di coesione funzionale, assicurando che funzioni correlate fossero raggruppate per formare moduli coesi, migliorando così manutenibilità e testabilità del sistema. E’ stato quindi eseguito un lavoro di analisi e decomposizione del problema seguendo delle euristiche di early design, riunendo gli use cases in gruppi a cui potessero essere associati dei subsystem ben delineati. L’applicazione delle euristiche assume che la progettazione della soluzione si baserà su un’architettura a microservizi.
\clearpage
