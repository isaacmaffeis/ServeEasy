\section{Introduzione}
Il sistema che si intende realizzare per il caso di studio è un software gestionale per ottimizzare la gestione delle comande di un ristorante, migliorando l’esperienza dei clienti, la produttività della cucina e l’efficacia della cassa.
Il sistema si baserà sull’utilizzo di tablet, che permettono ai commensali di ordinare i piatti desiderati, inserendo eventuali note, visualizzando lo stato degli ordini e richiedere il conto in modo semplice e veloce.
La cucina riceve le comande tramite una dashboard dedicata, che le ordina secondo un algoritmo di priorità basato su diversi parametri, come il tempo trascorso dall'ordinazione, la volontà del cliente, la durata di preparazione del piatto e altri fattori. La cucina può anche notificare il completamento di un ordine, che verrà visualizzato sul tablet del tavolo corrispondente. 
L’operatore di cassa sarà in grado di visualizzare il sommario degli ordini e stampare a schermo una ricevuta al cliente.
L'amministratore del ristorante può personalizzare la configurazione delle sale e dei menu, registrare i tavoli e gli account, e visualizzare delle statistiche sulle ordinazioni effettuate. Il sistema offre anche delle funzionalità opzionali, come la possibilità di far arrivare i piatti tutti insieme al tavolo, di allegare note agli ordini in preparazione, chiedere il conto al tavolo.
Il sistema si propone quindi di rendere più agile e soddisfacente il servizio di ristorazione, sfruttando le potenzialità della tecnologia e gli alti rendimenti di un algoritmo apposito.
\clearpage